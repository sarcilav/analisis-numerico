La complejidad de esta versi\'on esta dada por $O(n)$,porque en el \'unico lugar
en donde hay al menos dos ciclos anidados es en la parte\\
\vdots
\begin{verbatim}
    for ( i = 0; i < n ; i++)
      for ( j = 1; j <= STEPS; j++)
\end{verbatim}
\vdots
Y adem\'as no hay recursividad, ni operaciones muy complejas, que superen simples
y definidas operaciones aritm\'eticas.\\
Por lo que es un m\'etodo muy bueno para construir splines cerrados,
sin importar sin son uniformes o no, y adem\'as sin utilizar trucos
matem\'aticos ni programaci\'on avanzados. Por lo cual es el m\'etodo
que usaremos para construir una soluci\'on a nuestro problema.

