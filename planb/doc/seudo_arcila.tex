Una buena aproximaci\'on a este problema puede ser una
mezcla de los splines lineales, con los c\'ubicos y los b-splines.
La idea consiste en manejar la curva cerrada como un pol\'igono, de lados
muy peque\~nos, esto con el fin de al menos garantizar que el spline
toque los puntos iniciales, claramente esta es idea esta dada por los
splines lineales. Ahora bien entre cada uno de los puntos
adyacentes construir $recticas$ definidas como pasos sobra la
curva c\'ubica que brindan los splines c\'ubicos, y al final como determinar
esos pasos nos lo brinda los b-splines, como par\'ametros,
para curvas par\'ametricas para $x$ y $y$ respectivamente.
\begin{verbatim}
calcular_cubicas(n,  x[])
    w [n+1]
    v [n+1]
    y [n+1]
    D [n+1]
    z, F, G, H
    k
    /* Hay que resolver el siguiente sistema
       [4 1      1] [D[0]]   [3(x[1] - x[n])  ]
       |1 4 1     | |D[1]|   |3(x[2] - x[0])  |
       |  1 4 1   | | .  | = |      .         |
       |    ..... | | .  |   |      .         |
       |     1 4 1| | .  |   |3(x[n] - x[n-2])|
       [1      1 4] [D[n]]   [3(x[0] - x[n-1])]

       Descomponiendo las matrices:
        -- En triangular superior
        -- Triangular inferior
       Y con una sustitucion regresiva.
       Obtenmos D que son las derivadas
       de los puntos internos.
       */
    w[1] = v[1] = z = 1.0f/4.0f
    y[0] = z * 3 * (x[1] - x[n])
    H = 4
    F = 3 * (x[0] - x[n-1])
    G = 1
    for ( k = 1; k < n; k++)
      v[k+1] = z = 1/(4 - v[k])
      w[k+1] = -z * w[k]
      y[k] = z * (3*(x[k+1]-x[k-1]) - y[k-1])
      H = H - G * w[k]
      F = F - G * y[k-1]
      G = -v[k] * G
    endfor
    H = H - (G+1)*(v[n]+w[n])
    y[n] = F - (G+1)*y[n-1]

    D[n] = y[n]/H
    D[n-1] = y[n-1] - (v[n]+w[n])*D[n]
    for ( k = n-2; k >= 0; k--)
      D[k] = y[k] - v[k+1]*D[k+1] - w[k+1]*D[n]
    enfor
    Cubic C [n+1] /*Cubic es una clase o estructura para manejar las ecuaciones
     cubicas
    */
    for ( k = 0; k < n; k++)
      C[k] (float)x[k], D[k], 3*(x[k+1] - x[k]) - 2*D[k] - D[k+1],
                       2*(x[k] - x[k+1]) + D[k] + D[k+1])
    endfor
    C[n] Cubic((float)x[n], D[n], 3*(x[0] - x[n]) - 2*D[n] - D[0],
                     2*(x[n] - x[0]) + D[n] + D[0])
    return C

\end{verbatim}
Este c\'odigo que mostramos anteriormente nos sirve para obtener las curvas c\'ubicas
par\'ametricas, de tal forma que $f_x(t) = at^3+bt^2+ct+d$ y $f_y(t) = et^3+ft^2+gt+h$.
Con el m\'etodo para obtener las curvas c\'ubicas ya si podremos mostrar
como ser\'ia el resto de la idea.
\begin{verbatim}
    steps = 12
    Cubic X = calcular_cubicas(n, points.x)
    Cubic Y = calcular_cubicas(n, points.y)
    Polygon p
    p.add_point(X[0].eval(0),Y[0].eval(0) /*eval es la funcion que se
     encarga de evaluar el polinomio cubico en un valor determinado
    */
    for ( i = 0; i < n ; i++)
      for ( j = 1; j <= STEPS; j++)
        u = j / (float) STEPS
        p.addPoint(X[i].eval(u),Y[i].eval(u))
      endfor
    endfor
\end{verbatim}


